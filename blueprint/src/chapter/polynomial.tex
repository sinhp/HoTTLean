In this section we develop some of the definitions
and lemmas related to polynomial endofunctors that we will use
in the rest of the notes.

\medskip

\begin{defn}[Polynomial endofunctor]
  \label{defn:UvPoly} \lean{UvPoly} \leanok
  Let $\catC$ be a locally Cartesian closed category
  (in our case, presheaves on the category of contexts).
  This means for each morphism $t : B \to A$ we have an adjoint triple
  % https://q.uiver.app/#q=WzAsMyxbMCwyXSxbMSwyLCJcXGNhdEMgLyBBIl0sWzEsMCwiXFxjYXRDIC8gQiJdLFsyLDEsInRfKiIsMCx7Im9mZnNldCI6LTV9XSxbMiwxLCJ0XyEiLDIseyJvZmZzZXQiOjV9XSxbMSwyLCJ0XioiLDFdLFs1LDMsIiIsMix7ImxldmVsIjoxLCJzdHlsZSI6eyJuYW1lIjoiYWRqdW5jdGlvbiJ9fV0sWzQsNSwiIiwyLHsibGV2ZWwiOjEsInN0eWxlIjp7Im5hbWUiOiJhZGp1bmN0aW9uIn19XV0=
  \begin{center}\begin{tikzcd}
    & {\catC / B} \\
    \\
    {} & {\catC / A}
    \arrow[""{name=0, anchor=center, inner sep=0}, "{t_*}", bend left, shift left=5, from=1-2, to=3-2]
    \arrow[""{name=1, anchor=center, inner sep=0}, "{t_!}"', bend right, shift right=5, from=1-2, to=3-2]
    \arrow[""{name=2, anchor=center, inner sep=0}, "{t^*}"{description}, from=3-2, to=1-2]
    \arrow["\dashv"{anchor=center}, draw=none, from=1, to=2]
    \arrow["\dashv"{anchor=center}, draw=none, from=2, to=0]
  \end{tikzcd}\end{center}
  where $t^{*}$ is pullback, and $t_{!}$ is composition with $t$.

  Let $t : B \to A$ be a morphism in $\catC$.
  Then define $\Poly{t} : \catC \to \catC$ be the composition
  \[
    \Poly{t} := A_{!} \circ t_{*} \circ B^{*}
    \quad \quad \quad
  \]
    % https://q.uiver.app/#q=WzAsNCxbMCwwLCJcXGNhdEMiXSxbMSwwLCJcXGNhdEMgLyBCIl0sWzIsMCwiXFxjYXRDIC8gQSJdLFszLDAsIlxcY2F0QyJdLFswLDEsIkJeKiJdLFsxLDIsInRfKiJdLFsyLDMsIkFfISJdXQ==
  \begin{center}\begin{tikzcd}
    \catC & {\catC / B} & {\catC / A} & \catC
    \arrow["{B^*}", from=1-1, to=1-2]
    \arrow["{t_*}", from=1-2, to=1-3]
    \arrow["{A_!}", from=1-3, to=1-4]
  \end{tikzcd}\end{center}
\end{defn}

\medskip

\begin{prop}[Characterising property of Polynomial Endofunctors]
  \label{prop:UvPoly.equiv} \lean{UvPoly.equiv} \leanok
  \uses{defn:UvPoly}
  The data of a map into the polynomial applied to an object in $\catC$
% https://q.uiver.app/#q=WzAsMixbMCwwLCJcXEdhIl0sWzEsMCwiXFxQb2x5e3R9IFkiXSxbMCwxXV0=
\begin{center}\begin{tikzcd}
	\Ga & {\Poly{t} Y}
	\arrow[from=1-1, to=1-2]
\end{tikzcd}\end{center}
  corresponds to a pair of morphisms
  % https://q.uiver.app/#q=WzAsMixbMCwwLCJCXyEgdF4qXFxhbCJdLFsyLDAsIlkiXSxbMCwxLCJcXHRpbGRle1xcdGlsZGV7XFxwaGl9fSIsMCx7InN0eWxlIjp7ImJvZHkiOnsibmFtZSI6ImRhc2hlZCJ9fX1dXQ==
\[
    \al : \Ga \to A
    \quad \quad \text{ and }
    \quad \quad
    \be : \Ga \cdot \al \to Y
  \]
and this correspondance is natural in both $\Ga$ and $Y$.

Given any such $\phi$ we can extract $\al : \Ga \to A$
by composition
% https://q.uiver.app/#q=WzAsMyxbMCwwLCJcXEdhIl0sWzIsMCwiXFxQb2x5e3R9IFkiXSxbMSwxLCJBIl0sWzAsMSwiXFxwaGkiXSxbMSwyLCJ0XyogQl4qIFkiXSxbMCwyLCJcXGFsIiwyLHsic3R5bGUiOnsiYm9keSI6eyJuYW1lIjoiZGFzaGVkIn19fV1d
\begin{center}\begin{tikzcd}
	\Ga && {\Poly{t} Y} \\
	& A
	\arrow["\phi", from=1-1, to=1-3]
	\arrow["\al"', dashed, from=1-1, to=2-2]
	\arrow["{t_* B^* Y}", from=1-3, to=2-2]
\end{tikzcd}\end{center}
Applying the adjunction $t^{*} \dashv t_{*}$,
and viewing $\phi : \al \to t_* B^* Y$ as a map in the slice over $A$,
this corresponds to
  \[
    \al : \Ga \to A
    \quad \quad \text{ and }
    \quad \quad
  \]
  % https://q.uiver.app/#q=WzAsMyxbMCwwLCJCXyEgdF4qXFxhbCJdLFsyLDAsIkIgXFx0aW1lcyBZIl0sWzEsMSwiQiJdLFswLDEsIlxcdGlsZGV7XFxwaGl9IiwwLHsic3R5bGUiOnsiYm9keSI6eyJuYW1lIjoiZGFzaGVkIn19fV0sWzEsMiwiQl4qIFkiXSxbMCwyLCJ0XipcXGFsIiwyXV0=
  \begin{center}\begin{tikzcd}
    {B_! t^*\al} && {B \times Y} \\
    & B
    \arrow["{\tilde{\phi}}", dashed, from=1-1, to=1-3]
    \arrow["{t^*\al}"', from=1-1, to=2-2]
    \arrow["{B^* Y}", from=1-3, to=2-2]
  \end{tikzcd}\end{center}
  Applying the adjunction $B_{!} \dashv B^{*}$,
  this corresponds to
  % https://q.uiver.app/#q=WzAsMixbMCwwLCJCXyEgdF4qXFxhbCJdLFsyLDAsIlkiXSxbMCwxLCJcXHRpbGRle1xcdGlsZGV7XFxwaGl9fSIsMCx7InN0eWxlIjp7ImJvZHkiOnsibmFtZSI6ImRhc2hlZCJ9fX1dXQ==
\[
    \al : \Ga \to A
    \quad \quad \text{ and }
    \quad \quad
  \]
\begin{center}\begin{tikzcd}
    {\Ga \cdot \al := B_! t^*\al} && Y
    \arrow["{\beta}", dashed, from=1-1, to=1-3]
\end{tikzcd}\end{center}

  Henceforth we will write
  \[ (\al, \be) : \Ga \to \Poly{t} Y\]
  for this map,
  since it is uniquely determined by this data.
  
  This is natural in $\Ga$.
  Precomposition by $\si : \De \to \Ga$,
  acts on such a pair by
% https://q.uiver.app/#q=WzAsNCxbMiwxLCJcXFBvbHl7dH0gWSJdLFswLDFdLFsxLDEsIlxcR2EiXSxbMSwwLCJcXERlIl0sWzMsMiwiXFxzaSIsMl0sWzMsMCwiKFxcYWwgXFxjaXJjIFxcc2ksIFxcYmUgXFxjaXJjIHReKiBcXHNpKSJdLFsyLDAsIihcXGFsLCBcXGJlKSIsMl1d
\begin{center}\begin{tikzcd}
	& \De \\
	{} & \Ga & {\Poly{t} Y}
	\arrow["\si"', from=1-2, to=2-2]
	\arrow["{(\al \circ \si, \be \circ t^* \si)}", from=1-2, to=2-3]
	\arrow["{(\al, \be)}"', from=2-2, to=2-3]
\end{tikzcd}\end{center}
It is also natural in $f : X \to Y$,
meaning the morphism $\Poly{t}{f}$ acts on such a pair by
% https://q.uiver.app/#q=WzAsNCxbMiwwLCJcXFBvbHl7dH0gWCJdLFswLDBdLFsxLDAsIlxcR2EiXSxbMiwxLCJcXFBvbHl7dH17WX0iXSxbMiwwLCIoXFxhbCwgXFxiZSkiXSxbMCwzLCJcXFBvbHl7dH17Zn0iXSxbMiwzLCIoXFxhbCwgZiBcXGNpcmMgXFxiZSkiLDJdXQ==
\begin{center}\begin{tikzcd}
	{} & \Ga & {\Poly{t} X} \\
	&& {\Poly{t}{Y}}
	\arrow["{(\al, \be)}", from=1-2, to=1-3]
	\arrow["{(\al, f \circ \be)}"', from=1-2, to=2-3]
	\arrow["{\Poly{t}{f}}", from=1-3, to=2-3]
\end{tikzcd}\end{center}
\end{prop}

\begin{lemma}
  \label{lem:R}
  \uses{prop:UvPoly.equiv}
  Use $R$ to denote the fiber product
  \begin{center}
  % https://q.uiver.app/#q=WzAsNCxbMCwxLCJcXFBvbHl7dH0ge1l9Il0sWzEsMSwiQSJdLFswLDAsIlIiXSxbMSwwLCJCIl0sWzAsMSwidF8qIEJeKiBZIiwyXSxbMiwwLCJcXHJob197XFxQb2x5e319IiwyXSxbMiwzLCJcXHJob19cXFRlcm0iXSxbMywxLCJ0Il0sWzIsMSwiIiwxLHsic3R5bGUiOnsibmFtZSI6ImNvcm5lciJ9fV1d
\begin{tikzcd}
	R & B \\
	{\Poly{t} {Y}} & A
	\arrow["{\rho_\Term}", from=1-1, to=1-2]
	\arrow["{\rho_{\Poly{}}}"', from=1-1, to=2-1]
	\arrow["\lrcorner"{anchor=center, pos=0.125}, draw=none, from=1-1, to=2-2]
	\arrow["t", from=1-2, to=2-2]
	\arrow["{t_* B^* Y}"', from=2-1, to=2-2]
\end{tikzcd}
\end{center}

  By the universal property of pullbacks
  and \ref{prop:UvPoly.equiv},
  The data of a map $\Ga \to R$
  corresponds to the data of
  $\be : \Ga \to B$ and $(t \circ \be,y) : \Ga \to \Poly{t}{Y}$,
  or just $\be : \Ga \to B$ and $y : \Ga \cdot t \circ \be \to Y$
  % https://q.uiver.app/#q=WzAsNSxbMSwyLCJcXFBvbHl7dH0ge1l9Il0sWzIsMiwiQSJdLFsxLDEsIlIiXSxbMiwxLCJCIl0sWzAsMCwiXFxHYSJdLFswLDEsInRfKiBCXiogWSIsMl0sWzIsMCwiXFxyaG9fe1xcUG9seXt9fSIsMl0sWzIsMywiXFxyaG9fXFxUZXJtIiwxXSxbMywxLCJ0Il0sWzIsMSwiIiwxLHsic3R5bGUiOnsibmFtZSI6ImNvcm5lciJ9fV0sWzQsMywiXFxiZSJdLFs0LDIsIihcXGJlLCB5KSIsMV0sWzQsMCwiKHQgXFxjaXJjIFxcYmUseSkiLDJdXQ==
\begin{center}\begin{tikzcd}
	\Ga \\
	& R & B \\
	& {\Poly{t} {Y}} & A
	\arrow["{(\be, y)}"{description}, from=1-1, to=2-2]
	\arrow["{(t \circ \be,y)}"', bend right, from=1-1, to=3-2]
	\arrow["\be", bend left, from=1-1, to=2-3]
	\arrow["{\rho_{\Poly{}}}", from=2-2, to=3-2]
	\arrow["{\rho_B}"{description}, from=2-2, to=2-3]
	\arrow["\lrcorner"{anchor=center, pos=0.125}, draw=none, from=2-2, to=3-3]
	\arrow["{t_* B^* Y}", from=3-2, to=3-3]
	\arrow["t"', from=2-3, to=3-3]
\end{tikzcd}\end{center}

  By uniqueness in the universal property of pullbacks and \ref{prop:UvPoly.equiv},
  Precomposition by a map $\si : \De \to \Ga$ acts on such a pair by
% https://q.uiver.app/#q=WzAsNCxbMiwxLCJSIl0sWzAsMV0sWzEsMSwiXFxHYSJdLFsxLDAsIlxcRGUiXSxbMywyLCJcXHNpIiwyXSxbMywwLCIoXFxiZVxcY2lyYyBcXHNpLCB5IFxcY2lyYyB0XiogXFxzaSkiXSxbMiwwLCIoXFxiZSwgeSkiLDJdXQ==
\begin{center}\begin{tikzcd}
	& \De \\
	{} & \Ga & R
	\arrow["\si"', from=1-2, to=2-2]
	\arrow["{(\be\circ \si, y \circ t^* \si)}", from=1-2, to=2-3]
	\arrow["{(\be, y)}"', from=2-2, to=2-3]
\end{tikzcd}\end{center}
\end{lemma}

\medskip

\begin{defn}[Evaluation]
  \label{defn:UvPolyEv}
  \uses{defn:UvPoly}
  Let $\counit : \rho_B \to B \to B^* Y$
  denote the counit of the adjunction $f^* \dashv f_*$
  at the object $B^* Y$, recalling that
  $\rho_B = t^* t_* B^* Y$.
  Then viewing the object $B^* Y$ in the slice as the
  object $Y \times B$ in the ambient category,
  we define $\ev{}{} : R \to Y$ as the composition
  \begin{center}
  % https://q.uiver.app/#q=WzAsNCxbMSwwLCJZIFxcdGltZXMgQiJdLFswLDAsIlIiXSxbMCwxLCJCIl0sWzIsMCwiWSJdLFsxLDAsIlxcY291bml0Il0sWzEsMiwiXFxyaG9fQiIsMl0sWzAsMl0sWzAsMywiXFxwaV9ZIl0sWzEsMywiXFxldiIsMCx7ImN1cnZlIjotNH1dXQ==
\begin{tikzcd}
	R & {Y \times B} & Y \\
	B
	\arrow["\counit"', from=1-1, to=1-2]
	\arrow["\ev{}{}", bend left, from=1-1, to=1-3]
	\arrow["{\rho_B}"', from=1-1, to=2-1]
	\arrow["{\pi_Y}"', from=1-2, to=1-3]
	\arrow[from=1-2, to=2-1]
\end{tikzcd}
  \end{center}
\end{defn}
\medskip

\begin{lemma}[Evaluation Computation]
  \label{lem:UvPolyEvEq}
  \uses{defn:UvPolyEv}
  Suppose $(\be,y) : \Ga \to R$, as in \ref{lem:R}
  \[ \be : \Ga \to B \quad \text{ and } \quad y : \Ga \cdot t \circ \be \to Y\]
  Then the evaluation of $y$ at $\be$ can be computed as
  \[ \ev{}{} \circ (\be, y) = y \circ b \]
  where
% https://q.uiver.app/#q=WzAsNSxbMSwyLCJcXEdhIl0sWzIsMiwiQSJdLFsyLDEsIkIiXSxbMSwxLCJcXEdhIFxcY2RvdCB0IFxcY2lyYyBcXGJlIl0sWzAsMCwiXFxHYSJdLFswLDEsInQgXFxjaXJjIFxcYmUiLDJdLFsyLDEsInQiXSxbMywwXSxbMywyXSxbNCwzLCJiIiwxLHsic3R5bGUiOnsiYm9keSI6eyJuYW1lIjoiZGFzaGVkIn19fV0sWzQsMCwiIiwxLHsibGV2ZWwiOjIsInN0eWxlIjp7ImhlYWQiOnsibmFtZSI6Im5vbmUifX19XSxbNCwyLCJcXGJlIiwxXSxbMywxLCIiLDAseyJzdHlsZSI6eyJuYW1lIjoiY29ybmVyIn19XV0=
\begin{center}\begin{tikzcd}
	\Ga \\
	& {\Ga \cdot t \circ \be} & B \\
	& \Ga & A
	\arrow["b"{description}, dashed, from=1-1, to=2-2]
	\arrow["\be"{description}, bend left, from=1-1, to=2-3]
	\arrow[Rightarrow, bend right, no head, from=1-1, to=3-2]
	\arrow["v"', from=2-2, to=2-3]
	\arrow["d", from=2-2, to=3-2]
	\arrow["t", from=2-3, to=3-3]
	\arrow["{t \circ \be}"', from=3-2, to=3-3]
  \arrow["\lrcorner"{anchor=center, pos=0.125}, draw=none, from=2-2, to=3-3]
\end{tikzcd}\end{center}
  and
% https://q.uiver.app/#q=WzAsNyxbMywxLCJcXFBvbHl7dH0ge1l9Il0sWzMsMiwiQSJdLFsyLDEsIlIiXSxbMiwyLCJCIl0sWzEsMiwiWSBcXHRpbWVzIEIiXSxbMSwwLCJcXEdhIl0sWzAsMiwiWSJdLFswLDEsInRfKiBCXiogWSJdLFsyLDBdLFsyLDNdLFszLDEsInQiLDJdLFsyLDEsIiIsMSx7InN0eWxlIjp7Im5hbWUiOiJjb3JuZXIifX1dLFsyLDQsIlxcY291bml0IiwxXSxbNCwzLCJcXHBpX0IiLDJdLFs1LDAsIih0IFxcY2lyYyBcXGJlLHkpIiwxLHsiY3VydmUiOi0yfV0sWzUsMiwiKFxcYmUseSkiLDFdLFs1LDQsIih5IFxcY2lyYyBiLFxcYmUpIiwxLHsic3R5bGUiOnsiYm9keSI6eyJuYW1lIjoiZGFzaGVkIn19fV0sWzUsNiwieSBcXGNpcmMgYiIsMSx7ImN1cnZlIjoyLCJzdHlsZSI6eyJib2R5Ijp7Im5hbWUiOiJkYXNoZWQifX19XSxbNCw2LCJcXHBpX1kiXV0=
\begin{center}\begin{tikzcd}
	& \Ga \\
	&& R & {\Poly{t} {Y}} \\
	Y & {Y \times B} & B & A
	\arrow["{(\be,y)}"{description}, from=1-2, to=2-3]
	\arrow["{(t \circ \be,y)}"{description}, bend left, from=1-2, to=2-4]
	\arrow["{y \circ b}"{description}, bend right, dashed, from=1-2, to=3-1]
	\arrow["{(y \circ b,\be)}"{description}, dashed, from=1-2, to=3-2]
	\arrow[from=2-3, to=2-4]
	\arrow["\counit"{description}, from=2-3, to=3-2]
	\arrow[from=2-3, to=3-3]
	\arrow["\lrcorner"{anchor=center, pos=0.125}, draw=none, from=2-3, to=3-4]
	\arrow["{t_* B^* Y}", from=2-4, to=3-4]
	\arrow["{\pi_Y}", from=3-2, to=3-1]
	\arrow["{\pi_B}"', from=3-2, to=3-3]
	\arrow["t"', from=3-3, to=3-4]
\end{tikzcd}\end{center}
\end{lemma}
\begin{proof}
	It suffices to show
	$(\counit \circ (\be, y)) = (y \circ b , \be)$ instead.
	\begin{align*}
		  & \, \counit \circ (\be, y) \\
		= & \, \counit \circ (v \circ b, y \circ t^{*} d \circ t^{*} b)
		& \ref{evaluation_computation_diagram1} \\
		= & \, \counit \circ (v, y \circ t^{*} d) \circ b
		& \ref{lem:R},\ref{evaluation_computation_diagram2}\\
		= & \, \counit \circ t^{*} (t \circ \be, y) \circ b
		& \ref{evaluation_computation_diagram3}\\
		% = & \, \overline{\widetilde{\counit} \circ (t \circ \be, y)} \circ b \\
		= & \, \overline{(t \circ \be, y)} \circ b
    & \ref{evaluation_computation_diagram4}\\
		= & \, (y, v) \circ b
		& \ref{evaluation_computation_diagram5} \\
		= & \, (y \circ b, v \circ b)\\
		= & \, (y \circ b, \be)\\
	\end{align*}

\begin{figure}[h]
  \centering
  % https://q.uiver.app/#q=WzAsOCxbMSwxLCJcXEdhIFxcY2RvdCB0IFxcY2lyYyBcXGJlIl0sWzIsMSwiXFxHYSJdLFswLDEsIlxcR2EiXSxbMiwwLCJcXEdhIFxcY2RvdCB0IFxcY2lyYyBcXGJlIl0sWzEsMCwiXFxHYSBcXGNkb3QgdCBcXGNpcmMgXFxiZSBcXGNkb3QgdCBcXGNpcmMgXFxiZSJdLFszLDAsIkIiXSxbMywxLCJBIl0sWzAsMCwiXFxHYSBcXGNkb3QgdCBcXGNpcmMgXFxiZSJdLFswLDEsImQiLDJdLFsyLDAsImIiLDJdLFszLDEsImQiXSxbMiwxLCIiLDAseyJjdXJ2ZSI6MywibGV2ZWwiOjIsInN0eWxlIjp7ImhlYWQiOnsibmFtZSI6Im5vbmUifX19XSxbNCwwXSxbNCwzLCJ0XiogZCIsMCx7InN0eWxlIjp7ImJvZHkiOnsibmFtZSI6ImRhc2hlZCJ9fX1dLFszLDUsInYiXSxbNSw2LCJ0Il0sWzEsNiwidCBcXGNpcmMgXFxiZSIsMl0sWzcsNCwidF4qIGIiLDAseyJzdHlsZSI6eyJib2R5Ijp7Im5hbWUiOiJkYXNoZWQifX19XSxbNywyLCJkIl0sWzcsMywiIiwwLHsiY3VydmUiOi00LCJsZXZlbCI6Miwic3R5bGUiOnsiaGVhZCI6eyJuYW1lIjoibm9uZSJ9fX1dLFszLDYsIiIsMCx7InN0eWxlIjp7Im5hbWUiOiJjb3JuZXIifX1dLFs0LDEsIiIsMCx7InN0eWxlIjp7Im5hbWUiOiJjb3JuZXIifX1dLFs3LDAsIiIsMCx7InN0eWxlIjp7Im5hbWUiOiJjb3JuZXIifX1dXQ==
\begin{center}\begin{tikzcd}
	{\Ga \cdot t \circ \be} & {\Ga \cdot t \circ \be \cdot t \circ \be} & {\Ga \cdot t \circ \be} & B \\
	\Ga & {\Ga \cdot t \circ \be} & \Ga & A
	\arrow["{t^* b}", dashed, from=1-1, to=1-2]
	\arrow[bend left, Rightarrow, no head, from=1-1, to=1-3]
	\arrow[bend right, Rightarrow, no head, from=2-1, to=2-3]
	\arrow["d", from=1-1, to=2-1]
	\arrow["\lrcorner"{anchor=center, pos=0.125}, draw=none, from=1-1, to=2-2]
	\arrow["{t^* d}", dashed, from=1-2, to=1-3]
	\arrow[from=1-2, to=2-2]
	\arrow["\lrcorner"{anchor=center, pos=0.125}, draw=none, from=1-2, to=2-3]
	\arrow["v", from=1-3, to=1-4]
	\arrow["d", from=1-3, to=2-3]
	\arrow["\lrcorner"{anchor=center, pos=0.125}, draw=none, from=1-3, to=2-4]
	\arrow["t", from=1-4, to=2-4]
	\arrow["b"', from=2-1, to=2-2]
	\arrow["d"', from=2-2, to=2-3]
	\arrow["{t \circ \be}"', from=2-3, to=2-4]
\end{tikzcd}\end{center}
  \caption{$t^{*} d \circ t^{*} b = \id_{\Ga \cdot t \circ \be}$}
  \label{evaluation_computation_diagram1}
\end{figure}

\begin{figure}[h]
% https://q.uiver.app/#q=WzAsMyxbMCwwLCJcXEdhIl0sWzAsMSwiXFxHYSBcXGNkb3QgdCBcXGNpcmMgXFxiZSJdLFsxLDEsIlIiXSxbMCwxLCJiIiwyXSxbMSwyLCIodiwgeSBcXGNpcmMgdF57Kn0gZCkiLDJdLFswLDIsIih2IFxcY2lyYyBiLCB5IFxcY2lyYyB0XnsqfSBkIFxcY2lyYyB0XnsqfSBiKSJdXQ==
\begin{center}\begin{tikzcd}
	\Ga \\
	{\Ga \cdot t \circ \be} & R
	\arrow["b"', from=1-1, to=2-1]
	\arrow["{(v \circ b, y \circ t^{*} d \circ t^{*} b)}", from=1-1, to=2-2]
	\arrow["{(v, y \circ t^{*} d)}"', from=2-1, to=2-2]
\end{tikzcd}\end{center}
  \caption{$(v, y \circ t^{*} d) \circ b = (v \circ b, y \circ t^{*} d \circ t^{*} b)$}
  \label{evaluation_computation_diagram2}
\end{figure}

\begin{figure}[h]
% https://q.uiver.app/#q=WzAsNixbMSwwLCJSIl0sWzIsMCwiQiJdLFsxLDEsIlxcUG9seXt0fXtZfSJdLFsyLDEsIkEiXSxbMCwwLCJcXEdhIFxcY2RvdCB0IFxcY2lyYyBcXGJlIl0sWzAsMSwiXFxHYSJdLFswLDFdLFswLDJdLFsxLDMsInQiXSxbMiwzLCJ0XyogQl4qIFkiLDJdLFswLDMsIiIsMSx7InN0eWxlIjp7Im5hbWUiOiJjb3JuZXIifX1dLFs0LDAsIih2LHkgXFxjaXJjIHReKiBkKSIsMCx7InN0eWxlIjp7ImJvZHkiOnsibmFtZSI6ImRhc2hlZCJ9fX1dLFs0LDUsImQiLDFdLFs1LDIsIih0IFxcY2lyYyBcXGJlLCB5KSIsMl0sWzUsMywidCBcXGNpcmMgXFxiZSIsMSx7Im9mZnNldCI6MywiY3VydmUiOjN9XSxbNCwxLCJ2IiwxLHsib2Zmc2V0IjotMywiY3VydmUiOi0zfV0sWzQsMiwiIiwxLHsic3R5bGUiOnsibmFtZSI6ImNvcm5lciJ9fV1d
\begin{center}\begin{tikzcd}
	{\Ga \cdot t \circ \be} & R & B \\
	\Ga & {\Poly{t}{Y}} & A
	\arrow["{(v,y \circ t^* d)}", dashed, from=1-1, to=1-2]
	\arrow["v"{description}, shift left=3, bend left, from=1-1, to=1-3]
	\arrow["d"{description}, from=1-1, to=2-1]
	\arrow["\lrcorner"{anchor=center, pos=0.125}, draw=none, from=1-1, to=2-2]
	\arrow[from=1-2, to=1-3]
	\arrow[from=1-2, to=2-2]
	\arrow["\lrcorner"{anchor=center, pos=0.125}, draw=none, from=1-2, to=2-3]
	\arrow["t", from=1-3, to=2-3]
	\arrow["{(t \circ \be, y)}"', from=2-1, to=2-2]
	\arrow["{t \circ \be}"{description}, shift right=3, bend right, from=2-1, to=2-3]
	\arrow["{t_* B^* Y}"', from=2-2, to=2-3]
\end{tikzcd}\end{center}
% https://q.uiver.app/#q=WzAsMyxbMSwxLCJcXFBvbHl7dH17WX0iXSxbMCwxLCJcXEdhIl0sWzAsMCwiXFxHYSBcXGNkb3QgdCBcXGNpcmMgXFxiZSJdLFsxLDAsIih0IFxcY2lyYyBcXGJlLCB5KSIsMl0sWzIsMSwiZCIsMl0sWzIsMCwiKHQgXFxjaXJjIFxcYmUgXFxjaXJjIGQsIHkgXFxjaXJjIHReKiBkKSJdXQ==
\begin{center}\begin{tikzcd}
	{\Ga \cdot t \circ \be} \\
	\Ga & {\Poly{t}{Y}}
	\arrow["d"', from=1-1, to=2-1]
	\arrow["{(t \circ \be \circ d, y \circ t^* d)}", from=1-1, to=2-2]
	\arrow["{(t \circ \be, y)}"', from=2-1, to=2-2]
\end{tikzcd}\end{center}
using \ref{prop:UvPoly.equiv}, \ref{lem:R}

  \caption{$t^{*}(t \circ \be, y) = (v, y \circ t^{*}d)$}
	\label{evaluation_computation_diagram3}
\end{figure}

\begin{figure}[h]
% https://q.uiver.app/#q=WzAsOCxbMCwwLCJ0XioodCBcXGNpcmMgXFxiZSkiXSxbMCwxLCJ0XiogdF8qIEJeKiBZIl0sWzEsMSwiQl4qWSJdLFsyLDBdLFsyLDFdLFszLDAsInQgXFxjaXJjIFxcYmUiXSxbMywxLCJ0XyogQl4qIFkiXSxbNCwxLCJ0XyogQl4qWSJdLFswLDEsInReKiAodCBcXGNpcmMgXFxiZSwgeSkiLDJdLFsxLDIsIlxcY291bml0IiwyXSxbMCwyLCJcXG92ZXJsaW5leyh0IFxcY2lyYyBcXGJlLHkpfSIsMCx7InN0eWxlIjp7ImJvZHkiOnsibmFtZSI6ImRhc2hlZCJ9fX1dLFszLDQsIiIsMCx7ImxldmVsIjoyLCJzdHlsZSI6eyJoZWFkIjp7Im5hbWUiOiJub25lIn19fV0sWzYsNywiXFx3aWRldGlsZGV7XFxjb3VuaXR9IiwyLHsibGV2ZWwiOjIsInN0eWxlIjp7ImhlYWQiOnsibmFtZSI6Im5vbmUifX19XSxbNSw2LCIodCBcXGNpcmMgXFxiZSwgeSkiLDJdLFs1LDcsIih0IFxcY2lyYyBcXGJlLHkpIl1d
\begin{center}\begin{tikzcd}
	{t^*(t \circ \be)} && {} & {t \circ \be} \\
	{t^* t_* B^* Y} & {B^*Y} & { t^{*} \dashv t_{*}} & {t_* B^* Y} & {t_* B^*Y}
	\arrow["{t^* (t \circ \be, y)}"', from=1-1, to=2-1]
	\arrow["{\overline{(t \circ \be,y)}}", dashed, from=1-1, to=2-2]
	\arrow[Rightarrow, no head, from=1-3, to=2-3]
	\arrow["{(t \circ \be, y)}"', from=1-4, to=2-4]
	\arrow["{(t \circ \be,y)}", from=1-4, to=2-5]
	\arrow["\counit"', from=2-1, to=2-2]
	\arrow["{\widetilde{\counit}}"', Rightarrow, no head, from=2-4, to=2-5]
\end{tikzcd}\end{center}
  \caption{$\counit \circ t^{*} (t \circ \be, y) = \overline{(t \circ \be, y)}$}
	\label{evaluation_computation_diagram4}
\end{figure}

\begin{figure}[h]
% https://q.uiver.app/#q=WzAsOCxbMCwwLCJcXEdhIFxcY2RvdCB0IFxcY2lyYyBcXGJlIl0sWzAsMSwiQiJdLFsxLDAsIlkgXFx0aW1lcyBCIl0sWzIsMF0sWzIsMSwidF4qIFxcZGFzaHYgdF8qIl0sWzMsMCwiXFxHYSJdLFszLDEsIkEiXSxbNCwwLCJcXFBvbHl7dH17WX0iXSxbMCwxLCJ2ID0gdF4qICh0IFxcY2lyYyBcXGJlKSIsMl0sWzAsMiwiKHksdikiLDAseyJzdHlsZSI6eyJib2R5Ijp7Im5hbWUiOiJkYXNoZWQifX19XSxbMyw0LCIiLDAseyJsZXZlbCI6Miwic3R5bGUiOnsiaGVhZCI6eyJuYW1lIjoibm9uZSJ9fX1dLFs1LDYsInQgXFxjaXJjIFxcYmUiLDJdLFs1LDcsIih0IFxcY2lyYyBcXGJlLHkpIl0sWzcsNiwidF8qQl4qWSJdLFsyLDEsIkJeKlkiXV0=
\begin{center}\begin{tikzcd}
	{\Ga \cdot t \circ \be} & {Y \times B} & {} & \Ga & {\Poly{t}{Y}} \\
	B && {t^* \dashv t_*} & A
	\arrow["{(y,v)}", dashed, from=1-1, to=1-2]
	\arrow["{v = t^* (t \circ \be)}"', from=1-1, to=2-1]
	\arrow["{B^*Y}", from=1-2, to=2-1]
	\arrow[Rightarrow, no head, from=1-3, to=2-3]
	\arrow["{(t \circ \be,y)}", from=1-4, to=1-5]
	\arrow["{t \circ \be}"', from=1-4, to=2-4]
	\arrow["{t_*B^*Y}", from=1-5, to=2-4]
\end{tikzcd}\end{center}
	\caption{$\overline{(t \circ \be, y)} = (y,v)$}
	\label{evaluation_computation_diagram5}
\end{figure}

\end{proof}

\medskip

\begin{defn}[Polynomial composition]
  \label{defn:PolynomialComposition}
  \lean{UvPoly.comp}
  Let $f : B \to A$ and $g : D \to C$.
  Define the \emph{polynomial composition}
  $f \PolyComp g : Q \to \Poly{f}{C}$
  as the composition of the two
  vertical maps in the following
  \begin{center}
% https://q.uiver.app/#q=WzAsNyxbMCwwLCJEIl0sWzEsMCwiUSJdLFsxLDEsIlIiXSxbMCwxLCJDIl0sWzIsMSwiQiJdLFsyLDIsIkEiXSxbMSwyLCJcXFBvbHl7Zn17Q30iXSxbMiw2XSxbMiw0XSxbNCw1LCJmIl0sWzYsNSwiZl8qQl4qQyIsMl0sWzIsNSwiIiwxLHsic3R5bGUiOnsibmFtZSI6ImNvcm5lciJ9fV0sWzIsMywiXFxldnt9e30iXSxbMCwzLCJnIiwyXSxbMSwwXSxbMSwyXSxbMSwzLCIiLDEseyJzdHlsZSI6eyJuYW1lIjoiY29ybmVyIn19XSxbMSw2LCJmIFxcUG9seUNvbXAgZyIsMCx7ImxhYmVsX3Bvc2l0aW9uIjoxMCwiY3VydmUiOi0zLCJzdHlsZSI6eyJib2R5Ijp7Im5hbWUiOiJkYXNoZWQifX19XV0=
\begin{tikzcd}
	D & Q \\
	C & R & B \\
	& {\Poly{f}{C}} & A
	\arrow["g"', from=1-1, to=2-1]
	\arrow[from=1-2, to=1-1]
	\arrow["\lrcorner"{anchor=center, pos=0.125, rotate=-90}, draw=none, from=1-2, to=2-1]
	\arrow[from=1-2, to=2-2]
	\arrow["{f \PolyComp g}"{pos=0.1}, bend left, dashed, from=1-2, to=3-2]
	\arrow["{\ev{}{}}", from=2-2, to=2-1]
	\arrow[from=2-2, to=2-3]
	\arrow[from=2-2, to=3-2]
	\arrow["\lrcorner"{anchor=center, pos=0.125}, draw=none, from=2-2, to=3-3]
	\arrow["f", from=2-3, to=3-3]
	\arrow["{f_*B^*C}"', from=3-2, to=3-3]
\end{tikzcd}
  \end{center}

  Then the two functors
  \[
  \Poly{f \PolyComp g} \iso \Poly{f} \circ \Poly{g}
  \]
  are naturally isomorphic.
\end{defn}
\begin{proof}
  % TODO
\end{proof}

\medskip

\begin{defn}[Mate] \label{defn:CategoryTheory.Mate}
  \label{mate}
  Suppose
  % https://q.uiver.app/#q=WzAsMyxbMCwwLCJDIl0sWzEsMSwiQSJdLFsyLDAsIkIiXSxbMCwxLCJzIiwyXSxbMiwxLCJ0Il0sWzAsMiwiXFxyaG8iXV0=
\begin{center}\begin{tikzcd}
	C && B \\
	& A
	\arrow["\rho", from=1-1, to=1-3]
	\arrow["s"', from=1-1, to=2-2]
	\arrow["t", from=1-3, to=2-2]
\end{tikzcd}\end{center}
  Then we have a mate $\mu_{!} : \rho_{!} \circ s^{*} \Rightarrow t^{*}$.
  This is given by the universal property of pullbacks:
  given $f : x \to y$ in the slice $\catC / A$ we have
% https://q.uiver.app/#q=WzAsOSxbMiwxLCJZIl0sWzIsMCwiWCJdLFsyLDIsIkEiXSxbMSwyLCJCIl0sWzAsMiwiQyJdLFswLDEsIlxcYnVsbGV0Il0sWzEsMSwiXFxidWxsZXQiXSxbMSwwLCJcXGJ1bGxldCJdLFswLDAsIlxcYnVsbGV0Il0sWzEsMCwiZiJdLFswLDIsInkiXSxbMywyLCJ0IiwyXSxbNCwzLCJcXHJobyIsMl0sWzYsMF0sWzYsMywidF4qIHkiLDJdLFs2LDIsIiIsMix7InN0eWxlIjp7Im5hbWUiOiJjb3JuZXIifX1dLFs1LDQsInNeKiB5IiwyXSxbNSw2LCJcXG11X3sheX0iLDJdLFs1LDMsIiIsMix7InN0eWxlIjp7Im5hbWUiOiJjb3JuZXIifX1dLFs3LDYsInReKiBmIl0sWzcsMV0sWzcsMCwiIiwwLHsic3R5bGUiOnsibmFtZSI6ImNvcm5lciJ9fV0sWzgsNSwic14qIGYiLDJdLFs4LDcsIlxcbXVfeyF4fSJdLFs4LDYsIiIsMix7InN0eWxlIjp7Im5hbWUiOiJjb3JuZXIifX1dLFsxLDIsIngiLDAseyJjdXJ2ZSI6LTN9XSxbMjIsMTksIlxcbXVfISIsMSx7InNob3J0ZW4iOnsic291cmNlIjoyMCwidGFyZ2V0IjoyMH19XV0=
\begin{center}\begin{tikzcd}
	\bullet & \bullet & X \\
	\bullet & \bullet & Y \\
	C & B & A
	\arrow["{\mu_{!x}}", from=1-1, to=1-2]
	\arrow[""{name=0, anchor=center, inner sep=0}, "{s^* f}"', from=1-1, to=2-1]
	\arrow["\lrcorner"{anchor=center, pos=0.125}, draw=none, from=1-1, to=2-2]
	\arrow[from=1-2, to=1-3]
	\arrow[""{name=1, anchor=center, inner sep=0}, "{t^* f}", from=1-2, to=2-2]
	\arrow["\lrcorner"{anchor=center, pos=0.125}, draw=none, from=1-2, to=2-3]
	\arrow["f", from=1-3, to=2-3]
	\arrow["x", bend left = 70, from=1-3, to=3-3]
	\arrow["{\mu_{!y}}"', from=2-1, to=2-2]
	\arrow["{s^* y}"', from=2-1, to=3-1]
	\arrow["\lrcorner"{anchor=center, pos=0.125}, draw=none, from=2-1, to=3-2]
	\arrow[from=2-2, to=2-3]
	\arrow["{t^* y}"', from=2-2, to=3-2]
	\arrow["\lrcorner"{anchor=center, pos=0.125}, draw=none, from=2-2, to=3-3]
	\arrow["y", from=2-3, to=3-3]
	\arrow["\rho"', from=3-1, to=3-2]
	\arrow["t"', from=3-2, to=3-3]
	\arrow["{\mu_!}"{description}, shorten <=6pt, shorten >=6pt, Rightarrow, from=0, to=1]
\end{tikzcd}\end{center}
  By the calculus of mates we also have a reversed mate between the right adjoints
  $\mu^{*} : t_{*} \to s_{*} \circ \rho^{*}$.
  Explicitly $\mu^{*}$ is the composition
  % https://q.uiver.app/#q=WzAsNCxbMCwwLCJ0XyoiXSxbMSwwLCJzXyogXFxyaG9eKiBcXHJob18hIHNeKiB0XyoiXSxbMiwwLCJzXyogXFxyaG9eKiB0XiogdF8qIl0sWzMsMCwic18qIFxccmhvXioiXSxbMSwyLCJzXyogXFxyaG9eKiBcXG11XyEgdF8qIl0sWzAsMSwiXFx1bml0IHRfKiJdLFsyLDMsInNfKiBcXHJob14qIFxcY291bml0Il1d
\begin{center}\begin{tikzcd}
	{t_*} & {s_* \rho^* \rho_! s^* t_*} & {s_* \rho^* t^* t_*} & {s_* \rho^*}
	\arrow["{\unit t_*}", from=1-1, to=1-2]
	\arrow["{s_* \rho^* \mu_! t_*}", from=1-2, to=1-3]
	\arrow["{s_* \rho^* \counit}", from=1-3, to=1-4]
\end{tikzcd}\end{center}
\end{defn}

\medskip

\begin{defn}[Contravariant action of $\Poly{-}$ on a slice]
  \label{defn:UvPolySlice}
  Let $\Poly{-} : (\catC / A)^{\op} \to [\catC, \catC]$
  be defined by taking $s \mapsto \Poly{s}$ on objects and
  act on a morphism by
% https://q.uiver.app/#q=WzAsNyxbMiwyLCJDIl0sWzAsMSwiQSJdLFsyLDAsIkIiXSxbNCwxXSxbMywxXSxbNSwyLCJcXFBvbHl7c30iXSxbNSwwLCJcXFBvbHl7dH0iXSxbMCwxLCJzIl0sWzIsMSwidCIsMl0sWzAsMiwiXFxyaG8iLDJdLFs0LDMsIiIsMCx7InN0eWxlIjp7InRhaWwiOnsibmFtZSI6Im1hcHMgdG8ifX19XSxbNiw1LCJcXFN0YXJ7XFxyaG99Il1d
\begin{center}\begin{tikzcd}
	&& B &&& {\Poly{t}} \\
	A &&& {} & {} \\
	&& C &&& {\Poly{s}}
	\arrow["t"', from=1-3, to=2-1]
	\arrow["{\Star{\rho}}", from=1-6, to=3-6]
	\arrow[maps to, from=2-4, to=2-5]
	\arrow["\rho"', from=3-3, to=1-3]
	\arrow["s", from=3-3, to=2-1]
\end{tikzcd}\end{center}
where
\[\Star{\rho} := A_{!} (s_{*} \eta \circ \mu B^{*}) : \Poly{t} \to \Poly{s}\]
% https://q.uiver.app/#q=WzAsNSxbMCwwLCJcXGNhdEMiXSxbMCwxLCJcXGNhdEMgLyBDIl0sWzEsMSwiXFxjYXRDIC8gQiJdLFswLDIsIlxcY2F0QyAvIEEiXSxbMCwzLCJcXGNhdEMiXSxbMCwxLCJDXioiLDJdLFswLDIsIkJeKiJdLFsyLDEsIlxccmhvXioiLDFdLFsxLDMsInNfKiIsMl0sWzIsMywidF8qIl0sWzMsNCwiQV8hIl0sWzAsNCwiXFxQb2x5e3N9IiwyLHsiY3VydmUiOjV9XSxbMCw0LCJcXFBvbHl7dH0iLDAseyJvZmZzZXQiOi01LCJjdXJ2ZSI6LTV9XSxbMiw1LCJcXGV0YSIsMSx7InNob3J0ZW4iOnsic291cmNlIjozMCwidGFyZ2V0IjozMH19XSxbOSwxLCJcXG11IiwxLHsic2hvcnRlbiI6eyJzb3VyY2UiOjIwfX1dXQ==
\begin{center}\begin{tikzcd}[row sep = large]
	\catC \\
	{\catC / C} & {\catC / B} \\
	{\catC / A} \\
	\catC
	\arrow[""{name=0, anchor=center, inner sep=0}, "{C^*}"', from=1-1, to=2-1]
	\arrow["{B^*}", from=1-1, to=2-2]
	\arrow["{s_*}"', from=2-1, to=3-1]
	\arrow["{\rho^*}"{description}, from=2-2, to=2-1]
	\arrow[""{name=1, anchor=center, inner sep=0}, "{t_*}", from=2-2, to=3-1]
	\arrow["{A_!}", from=3-1, to=4-1]
	\arrow["\eta"{description}, shorten <=9pt, shorten >=9pt, Rightarrow, from=2-2, to=0]
	\arrow["\mu"{description}, shorten <=2pt, Rightarrow, from=1, to=2-1]
	\arrow["{\Poly{s}}"', bend right = 70, from=1-1, to=4-1]
	\arrow["{\Poly{t}}", bend left = 100, shift left = 15, from=1-1, to=4-1]
\end{tikzcd}\end{center}
  where $\mu = \mu^{*}$ is the mate from \ref{mate},
  and $\eta$ is the natural isomorphism given by pullback pasting.

  Pointwise, this natural transformation acts on a pair
  $(\al, \be) : \Ga \to \Poly{t}{X}$ by
% https://q.uiver.app/#q=WzAsNCxbMiwwLCJcXFBvbHl7dH0gWCJdLFswLDBdLFsxLDAsIlxcR2EiXSxbMiwxLCJcXFBvbHl7c317WH0iXSxbMiwwLCIoXFxhbCwgXFxiZSkiXSxbMCwzLCJcXFN0YXJ7XFxyaG99X1giXSxbMiwzLCIoXFxhbCwgXFxiZSBcXGNpcmMgXFxhbF4qIFxccmhvKSIsMl1d
\begin{center}\begin{tikzcd}
	{} & \Ga & {\Poly{t} X} \\
	&& {\Poly{s}{X}}
	\arrow["{(\al, \be)}", from=1-2, to=1-3]
	\arrow["{(\al, \be \circ \al^* \rho)}"', from=1-2, to=2-3]
	\arrow["{\Star{\rho}_X}", from=1-3, to=2-3]
\end{tikzcd}\end{center}
  where $\al^{*} \rho$ is defined as
% https://q.uiver.app/#q=WzAsNixbMCwyLCJcXEdhIl0sWzEsMiwiQSJdLFsxLDEsIkIiXSxbMSwwLCJDIl0sWzAsMCwiXFxHYSBcXGNkb3RfcyBcXGFsIl0sWzAsMSwiXFxHYSBcXGNkb3RfdCBcXGFsIl0sWzAsMSwiXFxhbCIsMl0sWzIsMSwidCJdLFszLDIsIlxccmhvIl0sWzUsMF0sWzUsMiwidF4qIFxcYWwiXSxbNSwxLCIiLDAseyJzdHlsZSI6eyJuYW1lIjoiY29ybmVyIn19XSxbNCwzLCJzXiogXFxhbCJdLFs0LDUsIlxcYWxeKiBcXHJobyIsMix7InN0eWxlIjp7ImJvZHkiOnsibmFtZSI6ImRhc2hlZCJ9fX1dLFs0LDIsIiIsMCx7InN0eWxlIjp7Im5hbWUiOiJjb3JuZXIifX1dXQ==
\begin{center}\begin{tikzcd}
	{\Ga \cdot_s \al} & C \\
	{\Ga \cdot_t \al} & B \\
	\Ga & A
	\arrow["{s^* \al}", from=1-1, to=1-2]
	\arrow["{\al^* \rho}"', dashed, from=1-1, to=2-1]
	\arrow["\lrcorner"{anchor=center, pos=0.125}, draw=none, from=1-1, to=2-2]
	\arrow["\rho", from=1-2, to=2-2]
	\arrow["{t^* \al}", from=2-1, to=2-2]
	\arrow[from=2-1, to=3-1]
	\arrow["\lrcorner"{anchor=center, pos=0.125}, draw=none, from=2-1, to=3-2]
	\arrow["t", from=2-2, to=3-2]
	\arrow["\al"', from=3-1, to=3-2]
\end{tikzcd}\end{center}
  We prove this now.
\end{defn}
\begin{proof}

  Firstly $\Star{\rho}_{X} = A_{!} (s_{*} \eta_{X} \circ \mu_{B^{*}X})$,
  so the first component $\al : \Ga \to A$
  is preserved by $\Star{\rho}_{X}$ and it suffices to show, in $\catC / A$
  % https://q.uiver.app/#q=WzAsNCxbMiwwLCJ0XyogQl4qIFgiXSxbMCwwXSxbMSwwLCJcXGFsIl0sWzIsMSwic18qIENeKiBYIl0sWzIsMCwiKFxcYWwsIFxcYmUpIl0sWzAsMywic197Kn0gXFxldGFfe1h9IFxcY2lyYyBcXG11X3tCXnsqfVh9Il0sWzIsMywiKFxcYWwsIFxcYmUgXFxjaXJjIFxcYWxeKiBcXHJobykiLDJdXQ==
\begin{center}\begin{tikzcd}
	{} & \al & {t_* B^* X} \\
	&& {s_* C^* X}
	\arrow["{(\al, \be)}", from=1-2, to=1-3]
	\arrow["{(\al, \be \circ \al^* \rho)}"', from=1-2, to=2-3]
	\arrow["{s_{*} \eta_{X} \circ \mu_{B^{*}X}}", from=1-3, to=2-3]
\end{tikzcd}\end{center}

  By the adjunction $s^{*} \dashv s_{*}$, it suffices to show, in $\catC / C$
% https://q.uiver.app/#q=WzAsNCxbMiwwLCJzXip0XyogQl4qIFgiXSxbMCwwXSxbMSwwLCJzXiogXFxhbCJdLFsyLDEsIkNeKiBYIl0sWzIsMCwic14qIChcXGFsLCBcXGJlKSJdLFswLDMsIlxcb3ZlcmxpbmV7c197Kn0gXFxldGFfe1h9IFxcY2lyYyBcXG11X3tCXnsqfVh9fSJdLFsyLDMsIlxcb3ZlcmxpbmV7KFxcYWwsIFxcYmUgXFxjaXJjIFxcYWxeKiBcXHJobyl9IiwyXV0=
\begin{center}\begin{tikzcd}
	{} & {s^* \al} & {s^*t_* B^* X} \\
	&& {C^* X}
	\arrow["{s^* (\al, \be)}", from=1-2, to=1-3]
	\arrow["{\overline{(\al, \be \circ \al^* \rho)}}"', from=1-2, to=2-3]
	\arrow["{\overline{s_{*} \eta_{X} \circ \mu_{B^{*}X}}}", from=1-3, to=2-3]
\end{tikzcd}\end{center}


Now we calculate $\overline{s_{*} \eta_{X} \circ \mu_{B^{*}X}} = \eta_{X} \circ \overline{\mu_{B^{*}X}}$.
So that our goal is to show
% https://q.uiver.app/#q=WzAsNSxbMiwwLCJzXip0XyogQl4qIFgiXSxbMCwwXSxbMSwwLCJzXiogXFxhbCJdLFsyLDEsIkNeKiBYIl0sWzMsMCwiXFxyaG9eKiBCXipYIl0sWzIsMCwic14qIChcXGFsLCBcXGJlKSJdLFsyLDMsIlxcb3ZlcmxpbmV7KFxcYWwsIFxcYmUgXFxjaXJjIFxcYWxeKiBcXHJobyl9IiwyXSxbMCw0LCJcXG92ZXJsaW5le1xcbXVfe0Jeeyp9WH19Il0sWzQsMywiXFxldGFfe1h9Il1d
\begin{center}\begin{tikzcd}
	{} & {s^* \al} & {s^*t_* B^* X} & {\rho^* B^*X} \\
	&& {C^* X}
	\arrow["{s^* (\al, \be)}", from=1-2, to=1-3]
	\arrow["{\overline{(\al, \be \circ \al^* \rho)}}"', from=1-2, to=2-3]
	\arrow["{\overline{\mu_{B^{*}X}}}", from=1-3, to=1-4]
	\arrow["{\eta_{X}}", "{\sim}"', outer sep = -1pt, from=1-4, to=2-3]
\end{tikzcd}\end{center}
Since $\eta_{X}$ is an isomorphism between two limits of the same diagram,
namely $X \times C \iso C_{!} C^{*} X \iso C_{!} \rho^{*} B^{*} X$,
it suffices to show that both $\overline{\mu_{B^{*}X}} \circ s^{*}(\al, \be)$
and $\overline{(\al, \be \circ \al^{*} \rho)}$ are uniquely determined by
the same two maps into $X$ and $C$.

By the characterising property of polynomial endofunctors (\ref{prop:UvPoly.equiv})
we calculate
\[ \overline{(\al, \be \circ \al^{*} \rho)} = (\be \circ \al^{*} \rho, s^{*} \al)\]
% https://q.uiver.app/#q=WzAsMTAsWzAsMSwiXFxhbCJdLFsxLDEsInNfKiBDXiogWCJdLFsyLDBdLFsyLDJdLFszLDEsInNeKiBcXGFsIl0sWzQsMSwiQ14qIFgiXSxbNSwwXSxbNSwyXSxbNiwxLCJDXyEgc14qIFxcYWwiXSxbNywxLCJYIl0sWzAsMSwiKFxcYWwsIFxcYmUgXFxjaXJjIFxcYWxeKiBcXHJobykiXSxbMiwzLCIiLDAseyJsZXZlbCI6Miwic3R5bGUiOnsiaGVhZCI6eyJuYW1lIjoibm9uZSJ9fX1dLFs0LDUsIlxcb3ZlcmxpbmV7KFxcYWwsXFxiZSBcXGNpcmMgXFxhbF4qIFxccmhvKX0iXSxbNCw1LCIoXFxiZSBcXGNpcmMgXFxhbF4qIFxccmhvLCBzXiogXFxhbCkiLDJdLFs2LDcsIiIsMix7ImxldmVsIjoyLCJzdHlsZSI6eyJoZWFkIjp7Im5hbWUiOiJub25lIn19fV0sWzgsOSwiXFxiZSBcXGNpcmMgXFxhbF4qIFxccmhvIiwyXV0=
\begin{center}\begin{tikzcd}
	&& {} &&& {} \\
	\al & {s_* C^* X} && {s^* \al} & {C^* X} && {C_! s^* \al} & X \\
	&& {} &&& {}
	\arrow[Rightarrow, no head, from=1-3, to=3-3]
	\arrow[Rightarrow, no head, from=1-6, to=3-6]
	\arrow["{(\al, \be \circ \al^* \rho)}", from=2-1, to=2-2]
	\arrow["{\overline{(\al,\be \circ \al^* \rho)}}", from=2-4, to=2-5]
	\arrow["{(\be \circ \al^* \rho, s^* \al)}"', from=2-4, to=2-5]
	\arrow["{\be \circ \al^* \rho}"', from=2-7, to=2-8]
\end{tikzcd}\end{center}
More formally, this means
$\be \circ \al^{*} \rho : C_{!} s^{*} \al \to X$ and
$s^{*} \al : C_{!} s^{*} \al \to C$ are the two maps that uniquely determine
the map
$C_{!} \overline{\al, \be \circ \al^{*} \rho} : C_{!} s^{*} \al \to X \times C$.

On the other hand,
% https://q.uiver.app/#q=WzAsMjMsWzEsMywic14qdF8qIEJeKiBYIl0sWzAsMywic14qIFxcYWwiXSxbNCwzLCJcXHJob14qIEJeKlgiXSxbMiwzLCJcXHJob14qIFxccmhvXyEgc14qdF8qIEJeKlgiXSxbMywzLCJcXHJob14qICB0XiogdF8qIEJeKlgiXSxbMCwwLCJcXGFsIl0sWzEsMCwidF8qIEJeKiBYIl0sWzIsMCwic18qIFxccmhvXiogXFxyaG9fISBzXip0XyogQl4qWCJdLFszLDAsInNfKiBcXHJob14qICB0XiogdF8qIEJeKlgiXSxbNCwwLCJzXyogXFxyaG8qIEJeKiBYIl0sWzAsMl0sWzQsMiwic14qIFxcZGFzaHYgc18qIl0sWzAsNV0sWzQsNSwiXFxyaG9fISBcXGRhc2h2IFxccmhvXioiXSxbMCw2LCJcXHJob18hIHNeKiBcXGFsIl0sWzEsNiwiXFxyaG9fISBzXip0XyogQl4qIFgiXSxbMiw2LCIgXFxyaG9fISBzXip0XyogQl4qWCJdLFszLDYsIiAgdF4qIHRfKiBCXipYIl0sWzQsNiwiIEJeKlgiXSxbMyw3LCJSIl0sWzQsNywiWCBcXHRpbWVzIEIiXSxbMSw3LCJTIl0sWzAsNywiXFxHYSBcXGNkb3RfcyBcXGFsIl0sWzEsMCwic14qIChcXGFsLCBcXGJlKSJdLFswLDIsIlxcb3ZlcmxpbmV7XFxtdV97Ql57Kn1YfX0iLDAseyJjdXJ2ZSI6NX1dLFs1LDYsIihcXGFsLCBcXGJlKSJdLFs2LDcsIlxcdW5pdF97dF8qIEJeKiBYfSJdLFs3LDgsInNfKiBcXHJob14qIFxcbXVfISB0XyogQl4qWCJdLFs4LDksInNfKiBcXHJob14qIFxcY291bml0X3tCXipYfSJdLFs2LDksIntcXG11X3tCXnsqfVh9fSIsMCx7ImN1cnZlIjo1fV0sWzAsMywiXFxvdmVybGluZXtcXHVuaXRfe3RfKiBCXiogWH19Il0sWzMsNCwiXFxyaG9eKiBcXG11XyEgdF8qIEJeKlgiXSxbNCwyLCJcXHJob14qIFxcY291bml0X3tCXipYfSJdLFsxMCwxMSwiIiwwLHsibGV2ZWwiOjIsInN0eWxlIjp7ImhlYWQiOnsibmFtZSI6Im5vbmUifX19XSxbMTIsMTMsIiIsMCx7ImxldmVsIjoyLCJzdHlsZSI6eyJoZWFkIjp7Im5hbWUiOiJub25lIn19fV0sWzE0LDE1LCJcXHJob18hIHNeKiAoXFxhbCwgXFxiZSkiXSxbMTUsMTYsIlxcb3ZlcmxpbmV7XFxvdmVybGluZXtcXHVuaXRfe3RfKiBCXiogWH19fSIsMCx7ImxldmVsIjoyLCJzdHlsZSI6eyJoZWFkIjp7Im5hbWUiOiJub25lIn19fV0sWzE2LDE3LCJcXG11XyEgdF8qIEJeKlgiXSxbMTcsMTgsIlxcY291bml0X3tCXipYfSJdLFsxNywxOSwiIiwwLHsibGV2ZWwiOjIsInN0eWxlIjp7ImhlYWQiOnsibmFtZSI6Im5vbmUifX19XSxbMTgsMjAsIiIsMCx7ImxldmVsIjoyLCJzdHlsZSI6eyJoZWFkIjp7Im5hbWUiOiJub25lIn19fV0sWzE1LDIxLCIiLDAseyJsZXZlbCI6Miwic3R5bGUiOnsiaGVhZCI6eyJuYW1lIjoibm9uZSJ9fX1dLFsxNCwyMiwiIiwyLHsibGV2ZWwiOjIsInN0eWxlIjp7ImhlYWQiOnsibmFtZSI6Im5vbmUifX19XV0=
\begin{center}\begin{tikzcd}
	\al & {t_* B^* X} & {s_* \rho^* \rho_! s^*t_* B^*X} & {s_* \rho^*  t^* t_* B^*X} & {s_* \rho* B^* X} \\
	\\
	{} &&&& {s^* \dashv s_*} \\
	{s^* \al} & {s^*t_* B^* X} & {\rho^* \rho_! s^*t_* B^*X} & {\rho^*  t^* t_* B^*X} & {\rho^* B^*X} \\
	\\
	{} &&&& {\rho_! \dashv \rho^*} \\
	{\rho_! s^* \al} & {\rho_! s^*t_* B^* X} & { \rho_! s^*t_* B^*X} & {  t^* t_* B^*X} & { B^*X} \\
	{\Ga \cdot_s \al} & S && R & {X \times B} \\
	\arrow["{(\al, \be)}", from=1-1, to=1-2]
	\arrow["{\unit_{t_* B^* X}}", from=1-2, to=1-3]
	\arrow["{{\mu_{B^{*}X}}}", bend right, from=1-2, to=1-5]
	\arrow["{s_* \rho^* \mu_! t_* B^*X}", outer sep = 5pt,  from=1-3, to=1-4]
	\arrow["{s_* \rho^* \counit_{B^*X}}", outer sep = 5pt, from=1-4, to=1-5]
	\arrow[Rightarrow, no head, from=3-1, to=3-5]
	\arrow["{s^* (\al, \be)}", from=4-1, to=4-2]
	\arrow["{\overline{\unit_{t_* B^* X}}}", from=4-2, to=4-3]
	\arrow["{\overline{\mu_{B^{*}X}}}", bend right, from=4-2, to=4-5]
	\arrow["{\rho^* \mu_! t_* B^*X}", outer sep = 5pt, from=4-3, to=4-4]
	\arrow["{\rho^* \counit_{B^*X}}", outer sep = 5pt, from=4-4, to=4-5]
	\arrow[Rightarrow, no head, from=6-1, to=6-5]
	\arrow["{\rho_! s^* (\al, \be)}", from=7-1, to=7-2]
	\arrow["{\overline{\overline{\unit_{t_* B^* X}}}}", outer sep = 5pt, Rightarrow, no head, from=7-2, to=7-3]
	\arrow["{\mu_! t_* B^*X}", from=7-3, to=7-4]
	\arrow["{\counit_{B^*X}}", from=7-4, to=7-5]
  \arrow[Rightarrow, no head, from=7-1, to=8-1]
  \arrow[Rightarrow, no head, from=7-2, to=8-2]
	\arrow[Rightarrow, no head, from=7-4, to=8-4]
	\arrow[Rightarrow, no head, from=7-5, to=8-5]
\end{tikzcd}\end{center}

  The mate $\mu_{!}$ is calculated via the universal map
  into the pullback $R$ (dotted below).
% https://q.uiver.app/#q=WzAsOSxbMiwyLCJBIl0sWzEsMiwiQiJdLFswLDIsIkMiXSxbMiwxLCJcXFBvbHl7dH0gWCJdLFsyLDAsIlxcR2EiXSxbMSwwLCJcXEdhIFxcY2RvdF90IFxcYWwiXSxbMSwxLCJSIl0sWzAsMSwiUyJdLFswLDAsIlxcR2EgXFxjZG90X3MgXFxhbCJdLFsxLDAsInQiLDJdLFsyLDEsIlxccmhvIiwyXSxbMywwLCJ0XyogQl4qIFgiXSxbNCwzLCIoXFxhbCwgXFxiZSkiXSxbNSw2XSxbNiwxLCJ0XiogdF8qIEJeKiBYIl0sWzYsM10sWzYsMCwiIiwxLHsic3R5bGUiOnsibmFtZSI6ImNvcm5lciJ9fV0sWzcsMiwic14qIHRfKkJeKiBYIiwyXSxbNyw2LCJcXG11XyEgdF8qIEJeKlgiLDIseyJzdHlsZSI6eyJib2R5Ijp7Im5hbWUiOiJkYXNoZWQifX19XSxbOCw3LCJzXiogKFxcYWwsIFxcYmUpIiwyXSxbOCw1XSxbNSw0XSxbNCwwLCJcXGFsIiwwLHsiY3VydmUiOi01fV0sWzgsMiwic14qIFxcYWwiLDIseyJjdXJ2ZSI6NX1dXQ==
\begin{center}\begin{tikzcd}
	{\Ga \cdot_s \al} & {\Ga \cdot_t \al} & \Ga \\
	S & R & {\Poly{t} X} \\
	\quad C \quad & B & A
	\arrow[from=1-1, to=1-2]
	\arrow["{s^* (\al, \be)}"', from=1-1, to=2-1]
	\arrow["{s^* \al}"', bend right = 70, shift right = 4, from=1-1, to=3-1]
	\arrow[from=1-2, to=1-3]
	\arrow[from=1-2, to=2-2]
	\arrow["{(\al, \be)}", from=1-3, to=2-3]
	\arrow["\al", bend left = 70, shift left = 4, from=1-3, to=3-3]
	\arrow["{\mu_! t_* B^*X}"', dashed, from=2-1, to=2-2]
	\arrow["{s^* t_*B^* X}"', from=2-1, to=3-1]
	\arrow[from=2-2, to=2-3]
	\arrow["{t^* t_* B^* X}", from=2-2, to=3-2]
	\arrow["\lrcorner"{anchor=center, pos=0.125}, draw=none, from=2-2, to=3-3]
	\arrow["{t_* B^* X}", from=2-3, to=3-3]
	\arrow["\rho"', from=3-1, to=3-2]
	\arrow["t"', from=3-2, to=3-3]
\end{tikzcd}\end{center}

  Using the characterization of maps into $R$ from \ref{lem:R}
  we can calculate
  \[ \mu_! t_* B^*X \circ s^* (\al, \be) =
    (\rho \circ s^{*} \al, \be \circ t^{*}\al^{*} s)\]
  since the first component is simply the map $\Ga \cdot_{s} \al \to B$
  and the second component is the second component of the map
  \[ (\al \circ \al^{*} s, \be \circ t^{*} \al^{*} s)
    = (\al, \be) \circ \al^{*} s : \Ga \cdot_{s} \al \to \Poly{t}{X}\]

  Then using \ref{lem:UvPolyEvEq}
  \begin{align} \label{polynomial_star_equation1}
    & \overline{\overline{\mu_{B^{*}X}} \circ s^{*}(\al, \be)} \\
    = \, & \counit_{B^{*} X} \circ \mu_! t_* B^*X \circ s^* (\al, \be)\\
    = \, & \counit_{B^{*} X} \circ (\rho \circ s^{*} \al, \be \circ t^{*}\al^{*} s)\\
    = \, & (\be \circ t^{*}\al^{*} s \circ r, \rho \circ s^{*} \al) \\
    = \, & (\be \circ \al^{*} \rho, \rho \circ s^{*} \al) \\
    : \, \, & \Ga \cdot_s \al \to X \times B
  \end{align}
  where
 % https://q.uiver.app/#q=WzAsNSxbMiwyLCJBIl0sWzEsMiwiXFxHYSBcXGNkb3RfcyBcXGFsIl0sWzIsMSwiQiJdLFsxLDEsIlxcR2EgXFxjZG90X3MgXFxhbCBcXGNkb3RfdCBcXGFsIFxcY2lyYyBcXGFsXipzIl0sWzAsMCwiXFxHYSBcXGNkb3RfcyBcXGFsIl0sWzIsMCwidCJdLFszLDFdLFs0LDMsInIiLDIseyJzdHlsZSI6eyJib2R5Ijp7Im5hbWUiOiJkYXNoZWQifX19XSxbNCwxLCIiLDIseyJjdXJ2ZSI6NSwibGV2ZWwiOjIsInN0eWxlIjp7ImhlYWQiOnsibmFtZSI6Im5vbmUifX19XSxbNCwyLCJcXHJobyBcXGNpcmMgc14qIFxcYWwiLDAseyJjdXJ2ZSI6LTN9XSxbMywyXSxbMSwwLCJcXGFsIFxcY2lyYyBcXGFsXiogcyIsMl1d
\begin{center}\begin{tikzcd}
	{\Ga \cdot_s \al} \\
	& {\Ga \cdot_s \al \cdot_t \al \circ \al^*s} & B \\
	& {\Ga \cdot_s \al} & A
	\arrow["r"', dashed, from=1-1, to=2-2]
	\arrow["{\rho \circ s^* \al}", bend left, from=1-1, to=2-3]
	\arrow[bend right, Rightarrow, no head, from=1-1, to=3-2]
	\arrow[from=2-2, to=2-3]
	\arrow[from=2-2, to=3-2]
	\arrow["t", from=2-3, to=3-3]
	\arrow["{\al \circ \al^* s}"', from=3-2, to=3-3]
\end{tikzcd}\end{center}
and
% https://q.uiver.app/#q=WzAsOSxbMiwyLCJBIl0sWzEsMiwiXFxHYSJdLFswLDIsIlxcR2EgXFxjZG90X3MgXFxhbCJdLFsyLDEsIkIiXSxbMiwwLCJDIl0sWzEsMCwiXFxHYSBcXGNkb3RfcyBcXGFsIl0sWzEsMSwiXFxHYSBcXGNkb3RfdCBcXGFsIl0sWzAsMSwiXFxHYSBcXGNkb3RfcyBcXGFsIFxcY2RvdF90IFxcYWwgXFxjaXJjIFxcYWxeKnMiXSxbMCwwLCJcXEdhIFxcY2RvdF9zIFxcYWwiXSxbMSwwLCJcXGFsIiwyXSxbMiwxLCJcXGFsKnMiLDJdLFszLDAsInQiXSxbNCwzLCJcXHJobyJdLFs1LDYsIlxcYWxeKiBcXHJobyJdLFs2LDFdLFs2LDNdLFs2LDAsIiIsMSx7InN0eWxlIjp7Im5hbWUiOiJjb3JuZXIifX1dLFs3LDJdLFs3LDYsInReKiBcXGFsXiogcyIsMix7InN0eWxlIjp7ImJvZHkiOnsibmFtZSI6ImRhc2hlZCJ9fX1dLFs4LDcsInIiLDIseyJzdHlsZSI6eyJib2R5Ijp7Im5hbWUiOiJkYXNoZWQifX19XSxbOCw1LCIiLDAseyJsZXZlbCI6Miwic3R5bGUiOnsiaGVhZCI6eyJuYW1lIjoibm9uZSJ9fX1dLFs1LDQsInNeKiBcXGFsIl0sWzQsMCwicyIsMCx7ImN1cnZlIjotNX1dLFs4LDIsIiIsMix7ImN1cnZlIjo1LCJsZXZlbCI6Miwic3R5bGUiOnsiaGVhZCI6eyJuYW1lIjoibm9uZSJ9fX1dLFs1LDMsIiIsMCx7InN0eWxlIjp7Im5hbWUiOiJjb3JuZXIifX1dLFs3LDEsIiIsMSx7InN0eWxlIjp7Im5hbWUiOiJjb3JuZXIifX1dXQ==
\begin{center}\begin{tikzcd}
	{\Ga \cdot_s \al} & {\Ga \cdot_s \al} & C \\
	{\Ga \cdot_s \al \cdot_t \al \circ \al^*s} & {\Ga \cdot_t \al} & B \\
	{\Ga \cdot_s \al} & \Ga & A
	\arrow[Rightarrow, no head, from=1-1, to=1-2]
	\arrow["r"', dashed, from=1-1, to=2-1]
	\arrow[bend right = 50, shift right = 7, Rightarrow, no head, from=1-1, to=3-1]
	\arrow["s", bend left = 50, from=1-3, to=3-3]
	\arrow["{s^* \al}", from=1-2, to=1-3]
	\arrow["{\al^* \rho}", from=1-2, to=2-2]
	\arrow["\rho", from=1-3, to=2-3]
	\arrow["{t^* \al^* s}"', dashed, from=2-1, to=2-2]
	\arrow[from=2-1, to=3-1]
	\arrow[from=2-2, to=2-3]
	\arrow[from=2-2, to=3-2]
	\arrow["t", from=2-3, to=3-3]
	\arrow["{\al*s}"', from=3-1, to=3-2]
	\arrow["\al"', from=3-2, to=3-3]
	\arrow["\lrcorner"{anchor=center, pos=0.125}, draw=none, from=1-2, to=2-3]
	\arrow["\lrcorner"{anchor=center, pos=0.125}, draw=none, from=2-1, to=3-2]
	\arrow["\lrcorner"{anchor=center, pos=0.125}, draw=none, from=2-2, to=3-3]
\end{tikzcd}\end{center}
  Moving back along the adjunction $\rho_{!} \dashv \rho^{*}$
  \ref{polynomial_star_equation1} tells us that
 % https://q.uiver.app/#q=WzAsOSxbMywxLCJDIl0sWzMsMiwiQiJdLFszLDMsIjEiXSxbMiwzLCJYIl0sWzIsMiwiWCBcXHRpbWVzIEIiXSxbMiwxLCJYIFxcdGltZXMgQyJdLFswLDAsIlxcR2EgXFxjZG90X3MgXFxhbCJdLFsxLDBdLFsxLDJdLFswLDEsIlxccmhvIl0sWzEsMl0sWzMsMl0sWzQsMV0sWzQsM10sWzUsMF0sWzUsNF0sWzUsMSwiIiwwLHsic3R5bGUiOnsibmFtZSI6ImNvcm5lciJ9fV0sWzQsMiwiIiwwLHsic3R5bGUiOnsibmFtZSI6ImNvcm5lciJ9fV0sWzYsMywiXFxiZSBcXGNpcmMgXFxhbF4qIFxccmhvIiwyLHsiY3VydmUiOjN9XSxbNiw0LCJcXG92ZXJsaW5le1xcb3ZlcmxpbmV7XFxtdV97Ql57Kn1YfX0gXFxjaXJjIHNeeyp9KFxcYWwsIFxcYmUpfSIsMSx7ImN1cnZlIjoxfV0sWzYsNSwiXFxvdmVybGluZXtcXG11X3tCXnsqfVh9fSBcXGNpcmMgc157Kn0oXFxhbCwgXFxiZSkiLDFdLFs2LDAsInNeKiBcXGFsIiwwLHsiY3VydmUiOi0yfV1d
\begin{center}\begin{tikzcd}
	{\Ga \cdot_s \al} & {} \\
	&& {X \times C} & C \\
	& {} & {X \times B} & B \\
	&& X & 1
	\arrow["{\overline{\mu_{B^{*}X}} \circ s^{*}(\al, \be)}"{description}, from=1-1, to=2-3]
	\arrow["{s^* \al}", bend left, from=1-1, to=2-4]
	\arrow["{\overline{\overline{\mu_{B^{*}X}} \circ s^{*}(\al, \be)}}"{description}, bend right, from=1-1, to=3-3]
	\arrow["{\be \circ \al^* \rho}"', bend right = 60, from=1-1, to=4-3]
	\arrow[from=2-3, to=2-4]
	\arrow[from=2-3, to=3-3]
	\arrow["\lrcorner"{anchor=center, pos=0.125}, draw=none, from=2-3, to=3-4]
	\arrow["\rho", from=2-4, to=3-4]
	\arrow[from=3-3, to=3-4]
	\arrow[from=3-3, to=4-3]
	\arrow["\lrcorner"{anchor=center, pos=0.125}, draw=none, from=3-3, to=4-4]
	\arrow[from=3-4, to=4-4]
	\arrow[from=4-3, to=4-4]
\end{tikzcd}\end{center}

  So that, as required, $\overline{\mu_{B^{*}X}} \circ s^{*}(\al, \be)$
  and $\overline{(\al, \be \circ \al^{*} \rho)}$ are uniquely determined by
  the same two maps into $X$ and $C$.
\end{proof}

\medskip

\begin{defn}[Covariant action of $\Poly{-}$ on a cartesian square]
  \label{defn:UvPolyCartesianAction}
  We can also view taking polynomial endofunctors as a covariant
  functor on the category of arrows with cartesian squares
  as morphisms
  \[\Poly{-} : \CartArr(\catC) \to [\catC, \catC]\]
  where the action on a cartesian square is
% https://q.uiver.app/#q=WzAsOCxbMCwwLCJcXGNhdEMiXSxbMCwxLCJcXGNhdEMgLyBEIl0sWzEsMSwiXFxjYXRDIC8gQiJdLFswLDIsIlxcY2F0QyAvIEMiXSxbMCwzLCJcXGNhdEMiXSxbMSwyLCJcXGNhdEMvQSJdLFsxLDAsIlxcY2F0QyJdLFsxLDMsIlxcY2F0QyJdLFswLDEsIkReKiIsMl0sWzIsMSwiXFxyaG9eKiIsMV0sWzEsMywic18qIiwyXSxbMCw0LCJcXFBvbHl7c30iLDIseyJjdXJ2ZSI6NX1dLFsyLDUsInRfKiJdLFsxLDUsIlxcbXVeeyotMX0iLDEseyJsZXZlbCI6Mn1dLFs1LDMsIlxcdGhldGFeKiJdLFszLDQsIkNfISIsMl0sWzQsNywiIiwwLHsibGV2ZWwiOjIsInN0eWxlIjp7ImhlYWQiOnsibmFtZSI6Im5vbmUifX19XSxbNiw3LCJcXFBvbHl7dH0iLDAseyJvZmZzZXQiOi01LCJjdXJ2ZSI6LTV9XSxbMCw2LCIiLDAseyJsZXZlbCI6Miwic3R5bGUiOnsiaGVhZCI6eyJuYW1lIjoibm9uZSJ9fX1dLFs1LDcsIkFfISJdLFszLDcsIlxcbXVfISIsMix7InNob3J0ZW4iOnsidGFyZ2V0IjoyMH0sImxldmVsIjoyfV0sWzYsMiwiQl4qIl0sWzAsMiwiXFxldGFeLTEiLDEseyJsZXZlbCI6Mn1dXQ==
\[\begin{tikzcd}
	\catC & \catC \\
	{\catC / D} & {\catC / B} \\
	{\catC / C} & {\catC/A} \\
	\catC & \catC
	\arrow[equals, from=1-1, to=1-2]
	\arrow["{D^*}"', from=1-1, to=2-1]
	\arrow["{\eta^-1}"{description}, Rightarrow, from=1-1, to=2-2]
	\arrow["{\Poly{s}}"', shift right=5, bend right, from=1-1, to=4-1]
	\arrow["{B^*}", from=1-2, to=2-2]
	\arrow["{\Poly{t}}", shift left=5, bend left, from=1-2, to=4-2]
	\arrow["{s_*}"', from=2-1, to=3-1]
	\arrow["{\mu^{*-1}}"{description}, Rightarrow, from=2-1, to=3-2]
	\arrow["{\rho^*}"{description}, from=2-2, to=2-1]
	\arrow["{t_*}", from=2-2, to=3-2]
	\arrow["{C_!}"', from=3-1, to=4-1]
	\arrow["{\mu_!}"', shorten >=5pt, Rightarrow, from=3-1, to=4-2]
	\arrow["{\theta^*}", from=3-2, to=3-1]
	\arrow["{A_!}", from=3-2, to=4-2]
	\arrow[equals, from=4-1, to=4-2]
\end{tikzcd}\]
  given by the whiskered natural transformations
% https://q.uiver.app/#q=WzAsOCxbMCwwLCJcXGNhdEMiXSxbMCwxLCJcXGNhdEMgLyBEIl0sWzEsMSwiXFxjYXRDIC8gQiJdLFswLDIsIlxcY2F0QyAvIEMiXSxbMCwzLCJcXGNhdEMiXSxbMSwyLCJcXGNhdEMvQSJdLFsxLDAsIlxcY2F0QyJdLFsxLDMsIlxcY2F0QyJdLFswLDEsIkNeKiIsMl0sWzIsMSwiXFxyaG9eKiIsMV0sWzEsMywic18qIiwyXSxbMCw0LCJcXFBvbHl7c30iLDIseyJjdXJ2ZSI6NX1dLFsyLDUsInRfKiJdLFsxLDUsIlxcbXVeeyotMX0iLDEseyJsZXZlbCI6Mn1dLFs1LDMsIlxcdGhldGFeKiJdLFszLDQsIkNfISIsMl0sWzQsNywiIiwwLHsibGV2ZWwiOjIsInN0eWxlIjp7ImhlYWQiOnsibmFtZSI6Im5vbmUifX19XSxbNiw3LCJcXFBvbHl7dH0iLDAseyJvZmZzZXQiOi01LCJjdXJ2ZSI6LTV9XSxbMCw2LCIiLDAseyJsZXZlbCI6Miwic3R5bGUiOnsiaGVhZCI6eyJuYW1lIjoibm9uZSJ9fX1dLFs1LDcsIkFfISJdLFszLDcsIlxcbXVfISIsMix7InNob3J0ZW4iOnsidGFyZ2V0IjoyMH0sImxldmVsIjoyfV0sWzYsMiwiQl4qIl0sWzAsMiwiXFxldGFeLTEiLDEseyJsZXZlbCI6Mn1dXQ==
\begin{center}\begin{tikzcd}
	\catC & \catC \\
	{\catC / D} & {\catC / B} \\
	{\catC / C} & {\catC/A} \\
	\catC & \catC
	\arrow[Rightarrow, no head, from=1-1, to=1-2]
	\arrow["{C^*}"', from=1-1, to=2-1]
	\arrow["{\eta^-1}"{description}, Rightarrow, from=1-1, to=2-2]
	\arrow["{\Poly{s}}"', shift right=5, bend right = 30, from=1-1, to=4-1]
	\arrow["{B^*}", from=1-2, to=2-2]
	\arrow["{\Poly{t}}", shift left=5, bend left = 30, from=1-2, to=4-2]
	\arrow["{s_*}"', from=2-1, to=3-1]
	\arrow["{\mu^{*-1}}"{description}, Rightarrow, from=2-1, to=3-2]
	\arrow["{\rho^*}"{description}, from=2-2, to=2-1]
	\arrow["{t_*}", from=2-2, to=3-2]
	\arrow["{C_!}"', from=3-1, to=4-1]
	\arrow["{\mu_!}"', shorten >=5pt, Rightarrow, from=3-1, to=4-2]
	\arrow["{\theta^*}", from=3-2, to=3-1]
	\arrow["{A_!}", from=3-2, to=4-2]
	\arrow[Rightarrow, no head, from=4-1, to=4-2]
\end{tikzcd}\end{center}

  Furthermore, the natural transformation $\Poly{\kappa}$ is cartesian.
  meaning each naturality square is a pullback square.
% https://q.uiver.app/#q=WzAsNSxbMCwxXSxbMywwLCJcXFBvbHl7dH1YIl0sWzIsMCwiXFxQb2x5e3N9WCJdLFsyLDEsIlxcUG9seXtzfVkiXSxbMywxLCJcXFBvbHl7dH1ZIl0sWzIsMSwie1xcUG9seXtcXGthcHBhfX1fWSJdLFszLDQsIntcXFBvbHl7XFxrYXBwYX19X1kiLDJdLFsyLDMsIlxcUG9seXtzfSBmIiwyXSxbMSw0LCJcXFBvbHl7c30gZiJdLFsyLDQsIiIsMSx7InN0eWxlIjp7Im5hbWUiOiJjb3JuZXIifX1dXQ==
\begin{center}\begin{tikzcd}
	&& {\Poly{s}X} & {\Poly{t}X} \\
	{} && {\Poly{s}Y} & {\Poly{t}Y}
	\arrow["{{\Poly{\kappa}}_Y}", from=1-3, to=1-4]
	\arrow["{\Poly{s} f}"', from=1-3, to=2-3]
	\arrow["\lrcorner"{anchor=center, pos=0.125}, draw=none, from=1-3, to=2-4]
	\arrow["{\Poly{s} f}", from=1-4, to=2-4]
	\arrow["{{\Poly{\kappa}}_Y}"', from=2-3, to=2-4]
\end{tikzcd}\end{center}

  The natural transformation $\Poly{\kappa}$ computes in the following way
% https://q.uiver.app/#q=WzAsMTAsWzUsMSwiXFxHYSJdLFs1LDIsIlxcUG9seXtzfSBYIl0sWzYsMiwiXFxQb2x5e3R9e1h9Il0sWzMsMSwiQiJdLFszLDIsIkEiXSxbMiwyLCJDIl0sWzIsMSwiRCJdLFsxLDIsIlxcR2EiXSxbMSwxLCJcXEdhIFxcY2RvdCBfcyBcXGFsIl0sWzAsMCwiXFxHYSBcXGNkb3QgX3QgXFx0aGV0YSBcXGNpcmMgXFxhbCJdLFswLDEsIihcXGFsLCBcXGJlKSIsMl0sWzEsMiwie1xcUG9seXtcXGthcHBhfX1fWCIsMl0sWzAsMiwiKFxcdGhldGEgXFxjaXJjIFxcYWwsIFxcYmUgXFxjaXJjIGkpIl0sWzMsNCwidCJdLFs1LDQsIlxcdGhldGEiLDJdLFs2LDUsInMiLDJdLFs2LDMsIlxccmhvIiwxXSxbNiw0LCIiLDEseyJzdHlsZSI6eyJuYW1lIjoiY29ybmVyIn19XSxbNyw1LCJcXGFsIiwyXSxbOCw3XSxbOCw2XSxbOSw3XSxbOSw4LCJpIiwxLHsic3R5bGUiOnsiYm9keSI6eyJuYW1lIjoiZGFzaGVkIn19fV0sWzksM10sWzgsNSwiIiwxLHsic3R5bGUiOnsibmFtZSI6ImNvcm5lciJ9fV1d
\begin{center}\begin{tikzcd}
	{\Ga \cdot _t \theta \circ \al} \\
	& {\Ga \cdot _s \al} & D & B && \Ga \\
	& \Ga & C & A && {\Poly{s} X} & {\Poly{t}{X}}
	\arrow["i"{description}, dashed, from=1-1, to=2-2]
	\arrow[from=1-1, to=2-4]
	\arrow[from=1-1, to=3-2]
	\arrow[from=2-2, to=2-3]
	\arrow[from=2-2, to=3-2]
	\arrow["\lrcorner"{anchor=center, pos=0.125}, draw=none, from=2-2, to=3-3]
	\arrow["\rho"{description}, from=2-3, to=2-4]
	\arrow["s"', from=2-3, to=3-3]
	\arrow["\lrcorner"{anchor=center, pos=0.125}, draw=none, from=2-3, to=3-4]
	\arrow["t", from=2-4, to=3-4]
	\arrow["{(\al, \be)}"', from=2-6, to=3-6]
	\arrow["{(\theta \circ \al, \be \circ i)}", from=2-6, to=3-7]
	\arrow["\al"', from=3-2, to=3-3]
	\arrow["\theta"', from=3-3, to=3-4]
	\arrow["{{\Poly{\kappa}}_X}"', from=3-6, to=3-7]
\end{tikzcd}\end{center}
  using the fact that $\Ga \cdot_{s} \al$ and $\Ga \cdot_{t} \theta \circ \al$
  are limits of the same diagram.
\end{defn}
\begin{proof}
  %TODO computation proof

  We can use the computation of ${\Poly{\kappa}}_{X}$ and $\Poly{s}{f}$
  to show that the natural transformation $\Poly{\kappa}$
  is cartesian.
  Essentially, the first component of a map $\Ga \to \Poly{s}{X}$ is
  determined by its composition with $\Poly{s}{f}$ and its second component
  is determined by its composition with ${\Poly{\kappa}}_{X}$.
\end{proof}

\medskip

\begin{cor}\label{prop:UvPolySliceCartesian}
  If we have
  % https://q.uiver.app/#q=WzAsNixbMCwxLCJEIl0sWzAsMiwiQyJdLFsxLDEsIkIiXSxbMSwyLCJBIl0sWzAsMCwiRCciXSxbMSwwLCJCJyJdLFswLDEsInFfMSJdLFswLDJdLFsyLDMsInFfMiIsMl0sWzEsMywiXFx0aGV0YSIsMl0sWzAsMywiIiwxLHsic3R5bGUiOnsibmFtZSI6ImNvcm5lciJ9fV0sWzQsMCwiXFxyaG9fMSJdLFs1LDIsIlxccmhvXzIiLDJdLFs0LDVdLFs0LDEsInFfMSciLDIseyJjdXJ2ZSI6M31dLFs1LDMsInFfMiciLDAseyJjdXJ2ZSI6LTN9XSxbNCwyLCIiLDIseyJzdHlsZSI6eyJuYW1lIjoiY29ybmVyIn19XV0=
\begin{center}\begin{tikzcd}
	{D'} & {B'} \\
	D & B \\
	C & A
	\arrow[from=1-1, to=1-2]
	\arrow["{\rho_1}", from=1-1, to=2-1]
	\arrow["\lrcorner"{anchor=center, pos=0.125}, draw=none, from=1-1, to=2-2]
	\arrow["{q_1'}"', bend right, from=1-1, to=3-1]
	\arrow["{\rho_2}"', from=1-2, to=2-2]
	\arrow["{q_2'}", bend left, from=1-2, to=3-2]
	\arrow[from=2-1, to=2-2]
	\arrow["{q_1}", from=2-1, to=3-1]
	\arrow["\lrcorner"{anchor=center, pos=0.125}, draw=none, from=2-1, to=3-2]
	\arrow["{q_2}"', from=2-2, to=3-2]
	\arrow["\theta"', from=3-1, to=3-2]
\end{tikzcd}\end{center}
  then the two possible ways of obtaining composing the
  covariant and contravariant actions of $\Poly{-}$
  form a (strictly commuting) pullback square in $[\catC, \catC]$.
  % https://q.uiver.app/#q=WzAsNCxbMCwwLCJcXFBvbHl7cV8xfSJdLFsxLDAsIlxcUG9seXtxXzJ9Il0sWzAsMSwiXFxQb2x5e3FfMSd9Il0sWzEsMSwiXFxQb2x5e3FfMid9Il0sWzAsMSwiXFxQb2x5e1xca2FwcGF9Il0sWzAsMiwiXFxTdGFye1xccmhvXzF9IiwyXSxbMiwzLCJcXFBvbHl7XFxrYXBwYSd9IiwyXSxbMSwzLCJcXFN0YXJ7XFxyaG9fMn0iXSxbMCwzLCIiLDEseyJzdHlsZSI6eyJuYW1lIjoiY29ybmVyIn19XV0=
\begin{center}\begin{tikzcd}
	{\Poly{q_1}} & {\Poly{q_2}} \\
	{\Poly{q_1'}} & {\Poly{q_2'}}
	\arrow["{\Poly{\kappa}}", from=1-1, to=1-2]
	\arrow["{\Star{\rho_1}}"', from=1-1, to=2-1]
	\arrow["\lrcorner"{anchor=center, pos=0.125}, draw=none, from=1-1, to=2-2]
	\arrow["{\Star{\rho_2}}", from=1-2, to=2-2]
	\arrow["{\Poly{\kappa'}}"', from=2-1, to=2-2]
\end{tikzcd}\end{center}
\end{cor}
\begin{proof}
  To check that it commutes and is a pullback,
  it suffices to do this pointwise, for some $X \in \catC$.
  Then we simply unfold the computation for each of $\Poly{\kappa}$
  and $\Star{\rho}$.
\end{proof}
